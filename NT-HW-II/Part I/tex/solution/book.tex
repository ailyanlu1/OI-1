\documentclass[a4paper]{article}
\usepackage{ctex}
%\usepackage[slantfont, boldfont, CJKtextspaces, CJKmathspaces]{xeCJK}
\usepackage{fontspec,xunicode,xltxtra}
\usepackage[pagestyles]{titlesec}
\usepackage{indentfirst}
\usepackage[top=1in,bottom=1in,left=1.25in,right=1.25in]{geometry}
\usepackage{amsmath}
%usepackage[usenames,dvipsnames]{color}
\usepackage[colorlinks,linkcolor=red,anchorcolor=Blue,citecolor=green]{hyperref}

%\setCJKmainfont[BoldFont={黑体}, ItalicFont={楷体_GB2312}]{宋体}
\setmainfont{Liberation Serif} 
\setsansfont{Liberation Sans} 
\setmonofont{文泉驿等宽微米黑}

%\punctstyle{kaiming} % 开明式标点格式 
\renewcommand{\today}{}

\begin{document}

\title{Book解题报告}
\author{成都七中\ \  王迪}
\maketitle
\tableofcontents

\newpage

\section{题目大意}
构造一个长度为$N$的整数数列$\{S_N\}$,满足:
\begin{itemize}
\item $S_1 = X$;
\item 对于$1 < i \le N$,$S_i - S_{i-1}$要么为$+A$,要么为$-B$,其中$A,B$均为正整数;
\item $\sum_{1 \le i \le N}{S_i} = M$。
\end{itemize}
\par
对于$40\%$的数据,$N \le 100$,$A,B \le 5$; \par
对于$60\%$的数据,$N \le 1000$; \par
对于$100\%$的数据,$N \le 10^5$。 \par
数据保证有解。

\section{算法分析}
由于题目只要求构造一个可行解,所以很容易直接想到各种各样的贪心算法,但我所知的一些贪心算法都是不正确的,而盲目的贪心也不能帮助我们认清题目的本质。 \par
所以我们考虑用数学语言来刻画这个问题: \par
令$D_i = S_{i + 1} - S_i$,其中$1 \le i < N$,那么要么$D_i = A$,要么$D_i = -B$。 \par
则$S_i = X + \sum_{1 \le j < i}{D_j}$。我们尝试用$D_i$表示$M$:
\[
M = \sum_{1 \le i \le N}(X + \sum_{1 \le j < i}{D_j})
\]
\par
令$T = M - N \cdot X$,我们考虑上式每一个$D_i$出现的次数,则有:
\[
T = \sum_{1 \le i < N}{D_i \times (N - i)}
\]
\par
此时,我们发现问题是把$1$到$N-1$的整数分成两个集合,一个集合中数之和乘以$A$,另一个集合中数之和乘以$B$,而这两个值的差为$T$。 \par
我们已经可以给出一些可以获得部分分的动态规划算法。 \par
因为对$1$到$N-1$的每个整数有两个选择,我们考虑进行“降维”操作,即通过一些假设增加我们的已知量。 \par
对所有$1 \le i < N$令$D_i = +A$,计$Q = A \cdot \sum_{1 \le i < N}(N - i) = A \cdot \frac{N(N - 1)}{2}$。 \par
考虑$Q$和$T$的差值,而对于$D_i$若改变它为$-B$则这个差值将减少$(A+B)(N-i)$。 \par
因为数据保证有解,故$Q-T$必为$(A+B)$的倍数,且记$P=\frac{Q-T}{A+B}$,有$0 \le P \le \frac{N(N - 1)}{2}$。 \par
于是问题变成了一个经典问题:从$1$到$N-1$中选择一些数,使得和为$P$。 \par
现在就可以比较轻松地设计算法了。

\section{$40\%$的算法}
因为$N \le 100$,而$P = O(N^2)$,所以我们可以设计一个动态规划算法。 \par
记$dp[i][j]$表示用$1$到$i$的数能否得到选一些和为$j$,则有$dp[i][j] = dp[i - 1][j]\ or\ dp[i - 1][j - i]$。 \par
时间复杂度$O(N^3)$,期望得分$40$分。

\section{$60\%$的算法}
直接的动态规划显然没有利用题目的特殊性:可用的数字是从$1$到$N-1$,具有连续性! \par
容易设计出一个贪心的算法:每次从未选数中选择最大的一个不超过$P$的数,添加进方案中,然后迭代进行下去。 \par
这个结论其实是显然的:
\begin{itemize}
\item 若$P \le N - 1$,则可以直接构造出;
\item 若$P \ge N$,则我们的过程可以这样表示:$P=(N-1)+(N-2)+\cdots+(N-k)+R$。
\begin{itemize}
\item 若$R < N - k$,则可以直接构造;
\item 否则,可以把$(N - K - 1)$添加进方案,然后迭代下去。因为$P \le \frac{N(N-1)}{2}$,所以一定是可以构造出来的。
\end{itemize}
\end{itemize}
\par
所以简单的两重循环,每次找到可用的最大数即可,时间复杂度$O(N^2)$,期望得分$60$分。

\section{$100\%$的算法}
由上面的分析可以发现,我们构造出的方案一定是从$N-1$开始递减连续的一段,再加上一个单独的值,所以完全可以$O(N)$实现。期望得分$100$分。

\end{document}
