\section{Book}
\subsection{题目描述}
{\itshape
Wayne喜欢看书,更喜欢买书。 \par
某天Wayne在当当网上买书,买了很多很多书。Wayne有一个奇怪的癖好,就是第一本书的价格必须恰为$X$,而之后买的每一本书,若是比上一本更昂贵,则价格最多多$A$元;若是比上一本更便宜,则价格最多少$B$元。 \par
Wayne心血来潮,一口气买了$N$本书,但他记不得每本书的价格了,只记得总价格是$M$。Wayne于是很想知道一种可能的书价分布。为了简化问题,我们假定书价的定义域是整数,且每本书与上一本书的价格差,要么恰为$+A$,要么恰为$-B$。 \par
}
\subsection{输入格式}
{\itshape
第一行一个正整数$N$。 \par
第二行四个整数依次是$X,A,B,M$。
}
\subsection{输出格式}
{\itshape
输出一行$N$个整数,用空格隔开。数据保证有解。
}
\subsection{样例输入}
{\tt
4 \par
10 1 2 37
}
\subsection{样例输出}
{\tt
10 11 9 7
}
\subsection{数据规模}
{\itshape
对于$5\%$的数据,满足$N = 1$。 \par
对于另外$25\%$的数据,满足$A = B = 1, N \le 100$。 \par
对于另外$10\%$的数据,满足$A,B \le 5, N \le 100$。 \par
对于另外$20\%$的数据,满足$N \le 1000$。 \par
对于$100\%$的数据,满足$1 \le A, B \le 10^6$,$|X| \le 10^6$,$N \le 10^5$,$M$可用带符号$64$位整型存储。
}
